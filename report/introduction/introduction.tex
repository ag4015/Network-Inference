
\chapter{Introduction}

The brain is a complex machine, it allows the human being to think, communicate and feel. It does so thanks to the billions of neurons that communicate in a dense network through synapses. However, little is known about how it works. By studying how the neurons structure to store and process information we can understand how the brain as a whole functions. This could have important applications in medicine for curing diseases such as Parkinson \cite{OldeDubbelinkKimT.E.2014Dbnt} and epilepsy \cite{PONTEN2007918}, and in machine learning for the development of more intelligent neural networks.\\

In order to infer the network structure of a set of neurons, they are treated as a diffusion network where electrical spikes increase the likelihood of connected neurons to spike and therefore transmit a signal that travels as if it were a disease. By evaluating the time of "infection", the relationship between two neurons can be probabilistically estimated. After computing the relationship between all of the neurons, an estimate of the topology of the network can be obtained. \\

Previous work on this topic \cite{pranav_report, alexandru2018estimating} evaluated the feasibility of using a maximum-likelihood estimator algorithm, NetRate \cite{rodriguez2011uncovering}, for the inference of the structure of biological neural networks. A network was simulated using the Izhikevic neuron model \cite{izhikevich2003simple} and the Brian simulator \cite{10.3389/neuro.01.026.2009}. The connections between the neurons were then estimated, compared to the original network and the performance of the algorithm was evaluated. Recent developments in technology now allow scientists to obtain individual neuron spike information from the brain tissue \cite{ito2016spontaneous, ito2014large, litke2004does}. This data is very useful and serves as a mean of evaluating the performance of the algorithm with real neurons. Moreover, this information can help in creating simulated networks that resemble more the real biological ones.


