
\chapter{Simulating a biological neural network}

The inference of networks using NetRate has been used only on simulations. A random structure is generated and the spikes simulated using the Brian simulator. This is very useful because it allows the possibility of comparing the inferred network to a ground truth and evaluating the performance of the algorithm. However, the goal is to be able to implement it on real biological networks whose topology could provide insight to scientists. With this arise many difficulties that need to be dealt with.\\

It is important to find a suitable dataset to analyse. It must either be made out of voltage readings from an array of sensors in a cluster of neurons or spike times and indices\footnote{Here, the index is the neuron number that generates a specific spike} of those spikes. 

\section{Mouse cortical neuron dataset}
\section{Differences between the real and simulated biological neural network}
\section{Input stimulus to the system}
\subsection{System with random spikes}
\subsection{System with randomly spiking clusters}
\section{Cascade generation}
\subsection{Method of maximum cascades}
\subsection{Method of maximum independence}
\subsection{Optimal cascade generation}
