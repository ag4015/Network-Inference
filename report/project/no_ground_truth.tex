
\chapter{Testing performance without a ground truth}


The analysis of NetRate's ability to infer the edges of a simulated network is based on the knowledge of a ground truth. The availability of such information allows the researcher to compare the structure of the inferred and original simulated networks. However, in a practical scenario, this is not possible because of the lack of such knowledge. \\ 

It is still of interest to have some sort of performance measure of how well the network has been inferred. Such a task would require splitting the recording into a training and testing set. The first one would be given to NetRate to obtain the adjacency matrix and the latter would be employed in the estimation of the performance of the algorithm on the training set. One disadvantage of this analysis comes from the fact that less data is given to NetRate and, therefore, its performance is hindered. In the following section, one such performance metric is proposed and discussed.\\

\section{Spike prediction performance metric}

A complete knowledge of connectivity of a network would allow a researcher to predict, up until a certain extent, the spiking activity of the system. This is because given the event of node \(j\) spiking and the knowledge of the directed connexions from this neuron, the membrane potential of these nodes can be computed. If any of them exceeds the membrane potential threshold, it will become unstable and, eventually, it will spike. The proposed metric of \textit{spike prediction} builds on this idea. \\

Given the inferred adjacency matrix, the number \(K\) of firing events during a recording window of length \(T\) and the neuron index \(j\) of the first spike; the task is to predict which neurons are the ones that spike during the observation window. The predicted neurons will be the ones whose connectivity weights are the highest to the source neuron \(j\) by a path of any length. The path is defined as the number of nodes that are travelled to get to a certain destination node. The accuracy is defined as the ratio of correctly predicted neuron indexes over \(K\). Neurons that spike twice are only taken into account once.\\

The introduction of the path into the prediction allows for the modelling of diffusion processes. This would not be possible if only the neurons that are directly connected to \(j\) were taken into account. Under this prediction model, a nested connexion whose path is of length equal or greater than 1 would be given preference over one of a shorter path length if the connectivity weight is larger. This is non ideal if the path length is too long, however, the probability of this occurring decreases significantly for every new step in the path. \\

As an example, let the connexions of a network of 5 neurons be defined by table \ref{tab:example_connexions_no_ground_truth} and let the firing indices in a period of observation be given by Eq.\ref{eq:example_indices_no_ground_truth}. Then, \(j=1\) and the predicted neurons to spike would lie in the following order: 2, 4, 3, 5. Since \(K=4\), only the first three are kept i.e., 2 ,4, 3. Using this prediction the accuracy is equal to 0.67. The use of this metric is flawed for \(K\approx N\), where \(N\) is the size of the network. This is because the chance of guessing correctly the neuron indices is high. However, it is used in this example for illustration purposes.

\begin{equation}
indices = \{1, 2, 4, 5\}
\label{eq:example_indices_no_ground_truth}
\end{equation}

\begin{table}[]
\centering
\begin{tabular}{|c|c|c|}
\hline
Source node & Target node & Weight \\ \hline
1           & 2           & 25     \\ \hline
1           & 3           & 5      \\ \hline
1           & 5           & 1      \\ \hline
2           & 4           & 10     \\ \hline
4           & 2           & 2      \\ \hline
\end{tabular}
\caption{Example of connexions of a 5 neuron network}
\label{tab:example_connexions_no_ground_truth}
\end{table}

In order to see what the practical values of this metric are, a network of 15 neurons is simulated for 1500 seconds and the resulting spiking data is split into 60\% training and 40\% testing. Using the knowledge of the ground truth network, the prediction accuracy is equal to 83.43\%. However, if the inferred network, with an inference accuracy of 70.83\%, is used instead, the prediction metric drops to 10.44\%. The reason for this drastic fall lies on the high MAE (more on this in chapter 6).
